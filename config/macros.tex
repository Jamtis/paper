% Generic
\setcommand{\romanNumbering}[1]{\textsuperscript{(\romannum{#1})}}

% Parenthesis
\setcommand{\left}{\mleft}
\setcommand{\right}{\mright}
\DeclareAutoPairedDelimiter{\floor}{\lfloor}{\rfloor}
\DeclareAutoPairedDelimiter{\ceil}{\lceil}{\rceil}
\DeclareAutoPairedDelimiter{\round}{\lfloor}{\rceil}
\DeclareAutoPairedDelimiter{\parr}{(}{)}
\DeclareAutoPairedDelimiter{\pars}{[}{]}
\DeclareAutoPairedDelimiter{\parc}{\{}{\}}
\DeclareAutoPairedDelimiter{\abs}{\vert}{\vert}
\DeclareAutoPairedDelimiter{\card}{\vert}{\vert}
\DeclareAutoPairedDelimiter{\length}{\vert}{\vert}
\DeclareAutoPairedDelimiter{\bracket}{\langle}{\rangle}
\DeclareAutoPairedDelimiter{\inprod}{\langle}{\rangle}
\DeclareAutoPairedDelimiter{\parity}{\langle}{\rangle}
\DeclareAutoPairedDelimiter{\at}{\ifdisplay.\else{}\fi}{\vert}
% https://tex.stackexchange.com/questions/45713/creating-a-large-such-that-symbol
\setcommand{\given}{\;\ifnum\currentgrouptype=16 \middle\fi\vert\;}
% https://tex.stackexchange.com/questions/182749/underbrace-matrix-inside-brackets
\setcommand{\smashedunderbrace}[2]{\smash{\underbrace{#1}_{#2}}\vphantom{#1}}
\setcommand{\smashedoverbrace}[2]{\smash{\overbrace{#1}^{#2}}\vphantom{#1}}

% Asymptotics
\DeclareMathOperator{\negl}{negl}
\DeclareMathOperator{\owhl}{owhl}
\DeclareMathOperator{\poly}{poly}
\DeclareMathOperator{\polylog}{polylog}
% \DeclareMathOperator{\superpoly}{superpoly}
\DeclareMathOperator{\strictlyFaster}{\omega}
\DeclareMathOperator{\faster}{\Omega}
\DeclareMathOperator{\equallyFast}{\Theta}
\DeclareMathOperator{\slower}{\mathcal{O}}
\DeclareMathOperator{\strictlySlower}{\mathrm{o}}

% Sets
\setcommand{\bit}{\parc{\false,\true}}
\setcommand{\bits}{\bit^*}
\setcommand{\true}{1}
\setcommand{\false}{0}
\setcommand{\emptyString}{\varepsilon}
\setcommand{\Z}{\mathbb{Z}}
\setcommand{\R}{\mathbb{R}}
\setcommand{\N}{\mathbb{N}}

% Operations
% \setcommand{\mod}{\operatorname{mod}}
% \setcommand{\concat}{\mathopen{}\operatorname{||}\mathclose{}}
% \setcommand{\d}{\mathrm{d}}
\setcommand{\xor}{\oplus}
\setcommand{\diff}{\mathrel{\triangle}}
\setcommand{\BigXOR}{\bigoplus}
\setcommand{\cupdot}{\mathbin{\mathaccent\cdot\cup}}
% \setcommand{\bigcupdot}{\mathop{\cdot\bigcup}}
\setcommand{\bigcupdot}{\mathop{\hspace{8.8pt}\cdot \hspace{-8.8pt}\bigcup\hspace{5pt}}}
% \DeclareMathOperator{\bigcupdot}{\hspace{8.8pt}\cdot \hspace{-8.8pt}\bigcup\limits}
\DeclareMathOperator*{\argmin}{argmin}
\DeclareMathOperator*{\argmax}{argmax}

% Relations
\setcommand{\isequal}{\overset{?}{=}}
% \setcommand{\implies}{\Rightarrow}
\setcommand{\nimplies}{\centernot\implies}
% \setcommand{\cind}{\overset{c}{\approx}}
\setcommand{\sind}{\overset{s}{\approx}}
%\setcommand{\UCRealize}{\overset{\mathrm{UC}}{\geq}}
\setcommand{\isomorphic}{\cong}
\setcommand{\pind}{\protect\mathpalette{\protect\independenT}{\perp}}
\def\independenT#1#2{\mathrel{\rlap{\(#1#2\)}\mkern2mu{#1#2}}}
\let\originalleq\leq
\setcommand{\leq}[1][]{\originalleq_{\mathsf{#1}}}
\setcommand{\l}[1][]{<_{\mathsf{#1}}}
\let\originalgeq\geq
\setcommand{\geq}[1][]{\originalgeq_{\mathsf{#1}}}
\setcommand{\g}[1][]{>_{\mathsf{#1}}}

% Randomness
\setcommand{\Pr}[2][]{\operatorname{Pr}_{#1}\pars{#2}}
\setcommand{\Ex}[2][]{\operatorname{Ex}_{#1}\pars{#2}}
\setcommand{\Var}[2][]{\operatorname{Var}_{#1}\pars{#2}}
\setcommand{\getsr}{\overset{\$}{\gets}}

% Symbols
\setcommand{\ihat}{{\hat{\iota}}}
\setcommand{\jhat}{{\hat{\jmath}}}
\setcommand{\H}{\mathcal{H}}
\setcommand{\frakH}{\mathfrak{H}}
\setcommand{\D}{\mathcal{D}}
\setcommand{\T}{\mathcal{T}}
\setcommand{\E}{\mathcal{E}}
\setcommand{\A}{\mathcal{A}}
\setcommand{\M}{\mathcal{M}}
\setcommand{\K}{\mathsf{K}}
\setcommand{\C}{\mathsf{C}}
\setcommand{\F}{\mathsf{F}}
\setcommand{\I}{\mathcal{I}}
\setcommand{\Sim}{\mathcal{S}}
\setcommand{\U}{\mathcal{U}}
\setcommand{\S}{\mathcal{S}}
\setcommand{\fraka}{\mathfrak{a}}
\setcommand{\frakb}{\mathfrak{b}}
\setcommand{\frakc}{\mathfrak{c}}
\setcommand{\frakd}{\mathfrak{d}}
\setcommand{\frake}{\mathfrak{e}}
\setcommand{\frakf}{\mathfrak{f}}
\setcommand{\frakg}{\mathfrak{g}}
\setcommand{\frakh}{\mathfrak{h}}
\setcommand{\fraki}{\mathfrak{i}}
\setcommand{\frakj}{\mathfrak{j}}
\setcommand{\frakk}{\mathfrak{k}}
\setcommand{\frakl}{\mathfrak{l}}
\setcommand{\frakm}{\mathfrak{m}}
\setcommand{\frakn}{\mathfrak{n}}
\setcommand{\frako}{\mathfrak{o}}
\setcommand{\frakp}{\mathfrak{p}}
\setcommand{\frakq}{\mathfrak{q}}
\setcommand{\frakr}{\mathfrak{r}}
\setcommand{\fraks}{\mathfrak{s}}
\setcommand{\frakt}{\mathfrak{t}}
\setcommand{\fraku}{\mathfrak{u}}
\setcommand{\frakv}{\mathfrak{v}}
\setcommand{\frakw}{\mathfrak{w}}
\setcommand{\frakx}{\mathfrak{x}}
\setcommand{\fraky}{\mathfrak{y}}
\setcommand{\frakz}{\mathfrak{z}}
\setcommand{\frakA}{\mathfrak{A}}
\setcommand{\frakB}{\mathfrak{B}}
\setcommand{\frakC}{\mathfrak{C}}
\setcommand{\frakD}{\mathfrak{D}}
\setcommand{\frakE}{\mathfrak{E}}
\setcommand{\frakF}{\mathfrak{F}}
\setcommand{\frakG}{\mathfrak{G}}
\setcommand{\frakH}{\mathfrak{H}}
\setcommand{\frakI}{\mathfrak{I}}
\setcommand{\frakJ}{\mathfrak{J}}
\setcommand{\frakK}{\mathfrak{K}}
\setcommand{\frakL}{\mathfrak{L}}
\setcommand{\frakM}{\mathfrak{M}}
\setcommand{\frakN}{\mathfrak{N}}
\setcommand{\frakO}{\mathfrak{O}}
\setcommand{\frakP}{\mathfrak{P}}
\setcommand{\frakQ}{\mathfrak{Q}}
\setcommand{\frakR}{\mathfrak{R}}
\setcommand{\frakS}{\mathfrak{S}}
\setcommand{\frakT}{\mathfrak{T}}
\setcommand{\frakU}{\mathfrak{U}}
\setcommand{\frakV}{\mathfrak{V}}
\setcommand{\frakW}{\mathfrak{W}}
\setcommand{\frakX}{\mathfrak{X}}
\setcommand{\frakY}{\mathfrak{Y}}
\setcommand{\frakZ}{\mathfrak{Z}}

% Complexity Classes
\setcommand{\Heur}{\mathsf{Heur}}
\setcommand{\Avg}{\mathsf{Avg}}
\setcommand{\DTIME}{\mathsf{DTIME}}
\setcommand{\BPTIME}{\mathsf{BPTIME}}
\setcommand{\RTIME}{\mathsf{RTIME}}
\setcommand{\ZPTIME}{\mathsf{ZPTIME}}
\setcommand{\NTIME}{\mathsf{NTIME}}
\setcommand{\SPACE}{\mathsf{SPACE}}
\setcommand{\PSPACE}{\mathsf{PSPACE}}
\setcommand{\LEARN}{\mathsf{LEARN}}
\setcommand{\NP}{\mathsf{NP}}
\setcommand{\NEXP}{\mathsf{NEXP}}
\setcommand{\EXP}{\mathsf{EXP}}
\setcommand{\BPP}{\mathsf{BPP}}
\setcommand{\Ppoly}{\mathsf{P/poly}}
\setcommand{\P}{\mathsf{P}}
\setcommand{\QP}{\mathsf{QP}}
\setcommand{\UP}{\mathsf{UP}}
\setcommand{\DSR}{\mathsf{DSR}}
\setcommand{\RE}{\mathsf{RE}}
\setcommand{\Rrecursive}{\mathsf{R}}