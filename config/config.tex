% https://tex.stackexchange.com/questions/128797/a-renewcommand-that-does-not-care-if-a-command-is-already-defined
\makeatletter
\newcommand*{\setcommand}{%
    \@star@or@long\set@command
}
\newcommand*{\set@command}[1]{%
    \provide@command{#1}{}%
    % \let#1\@empty % would be more efficient, but without error checking
    \renew@command{#1}%
}
\makeatother

%% geometry
\ifbeamer{}{
    \ifdraft
        \geometry{
            paperwidth=25cm,
            textheight=24cm,
            heightrounded,
            left=1cm,
            right=8cm,
            marginparwidth=7cm,
            asymmetric
        }
    \else
        \iffullversion
            \geometry{
                a4paper,
                left=1in,
                right=1in,
            }
        \fi
    \fi
}

%% Environments
\theoremstyle{plain}
\crefname{question}{Question}{Questions}
\declaretheorem[name=Notation]{notation}
\crefname{notation}{Notation}{Notations}
\declaretheorem[name=Observation]{observation}
\crefname{observation}{Observation}{Observations}
\let\claim\relax
\declaretheorem[name=Claim]{claim}
\crefname{claim}{Claim}{Claims}
\ifbeamer{}{
    \declaretheorem[name=Fact]{fact}
    \crefname{fact}{Fact}{Facts}
}
\declaretheorem[name=Assumption]{assumption}
\crefname{assumption}{Assumption}{Assumptions}
\declaretheorem[name=Construction]{construction}
\crefname{construction}{Construction}{Constructions}

\newenvironment{nalign}{
    \begin{equation}
        \begin{aligned}
            }{
        \end{aligned}
    \end{equation}
    \ignorespacesafterend
}

\newenvironment{sitemize}{
    \begin{itemize}[leftmargin=3.5mm]
        }{
    \end{itemize}
    \ignorespacesafterend
}

\newenvironment{senumerate}{
    \begin{enumerate}[leftmargin=3.5mm]
        }{
    \end{enumerate}
    \ignorespacesafterend
}

\makeatletter
\let\oldframetitle\mdf@@frametitle@use
\let\mdf@@frametitle@use\relax
\setcommand{\mdf@@frametitle@use}{}
\makeatother
\newcommand{\newcontdbox}[2]{
    \newenvironment{#1}[1]{
        \setcommand{\sitem}{}
        % \tolerance=6500
        \tolerance=99999
        \begin{mdframed}[
                frametitle={#2 ##1 (cont'd)},
                repeatframetitle=true,
                frametitlealignment=\center,
                frametitlebelowskip=0,
                frametitlefont=\textrm,
                innertopmargin=2mm,
                innerleftmargin=1mm,
                innerrightmargin=1mm,
                everyline=true,
                needspace=6cm
            ]
            \mdfsubtitle[
                subtitleaboveskip=0,
                subtitlebelowskip=0,
                subtitleinneraboveskip=0,
                subtitleinnerbelowskip=1mm,
                subtitlefont=\textrm
            ]{#2 ##1}
            }{
        \end{mdframed}
        \ignorespacesafterend
    }
}
\newcontdbox{functionalitybox}{Functionality}
\newcontdbox{protocolbox}{Protocol}
\newcontdbox{simulatorbox}{Simulator for}

% https://tex.stackexchange.com/questions/391726/the-quotation-environment
\NewDocumentCommand{\bywhom}{m}{% the Bourbaki trick
    {\nobreak\hfill\penalty50\hskip1em\null\nobreak
            \hfill\mbox{\normalfont{#1}}%
            \parfillskip=0pt \finalhyphendemerits=0 \par}%
}

\NewDocumentEnvironment{pquotation}{m}{
    \begin{quoting}[
            leftmargin=\parindent,
            rightmargin=\parindent,]\itshape
        }{
    \end{quoting}
    \vspace{-3mm}
    \bywhom{#1}
}

\NewDocumentEnvironment{hybrids}{}{
    \begin{enumerate}[label=\(\mathcal{H}_{\arabic*}\)]
        }{
    \end{enumerate}%
}

% https://tex.stackexchange.com/questions/223948/suggestions-for-nested-proofs
\newenvironment{subproof}[1][\proofname]{%
    \providecommand{\qedsymbol}{}%
    \renewcommand{\qedsymbol}{\(\blacksquare\)}%
    \begin{proof}[#1]%
        }{%
    \end{proof}%
}

\newenvironment{bralign}{
    \begingroup
    \allowdisplaybreaks
    \align
}{
    \endalign
    \endgroup
}

%% Square at end of each proof
\let\originalendproof\endproof
% \providecommand{\endproof}{~\hfill\originalendproof}

%% Todos
\ifdraft
\else
    \setcommand{\marginpar}[1]{}
    \setcommand{\marginnote}[1]{}
\fi
% fix margin are on the right side
% https://tex.stackexchange.com/questions/69595/marginnote-always-on-right-side-of-the-page
\makeatletter
\long\def\@mn@@@marginnote[#1]#2[#3]{%
  \begingroup
    \ifmmode\mn@strut\let\@tempa\mn@vadjust\else
      \if@inlabel\leavevmode\fi
      \ifhmode\mn@strut\let\@tempa\mn@vadjust\else\let\@tempa\mn@vlap\fi
    \fi
    \@tempa{%
      \vbox to\z@{%
        \vss
        \@mn@margintest
        \if@reversemargin\if@tempswa
            \@tempswafalse
          \else
            \@tempswatrue
        \fi\fi
        %\if@tempswa
          \rlap{%
            \if@mn@verbose
              \PackageInfo{marginnote}{xpos seems to be \@mn@currxpos}%
            \fi
            \begingroup
              \ifx\@mn@currxpos\relax\else\ifx\@mn@currxpos\@empty\else
                  \kern-\dimexpr\@mn@currxpos\relax
              \fi\fi
              \ifx\@mn@currpage\relax
                \let\@mn@currpage\@ne
              \fi
              \if@twoside\ifodd\@mn@currpage\relax
                  \kern\oddsidemargin
                \else
                  \kern\evensidemargin
                \fi
              \else
                \kern\oddsidemargin
              \fi
              \kern 1in
            \endgroup
            \kern\marginnotetextwidth\kern\marginparsep
            \vbox to\z@{\kern\marginnotevadjust\kern #3
              \vbox to\z@{%
                \hsize\marginparwidth
                \linewidth\hsize
                \kern-\parskip
                \marginfont\raggedrightmarginnote\strut\hspace{\z@}%
                \ignorespaces#2\endgraf
                \vss}%
              \vss}%
          }%
      }%
    }%
  \endgroup
}
\makeatother

%% lists
\setcommand\labelitemi{\textbullet}
\setcommand\labelitemii{\textopenbullet}
%\setcommand\labelitemi{$\vcenter{\hbox{\tiny$\bullet$}}$}
\setlist{noitemsep, topsep=0pt}
\setenumerate[2]{labelindent=0pt,itemindent=0pt}

%% tikz
\usetikzlibrary{calc}
\usetikzlibrary{patterns}
\usetikzlibrary{shapes.misc}
\usetikzlibrary{shapes.callouts}
\usetikzlibrary{tikzmark}
\usetikzlibrary{decorations.pathreplacing,calligraphy}
\tikzset{
    square/.style={rectangle, minimum width=.8cm, minimum height=.8cm, draw=black!60},
    % party/.style={rectangle, minimum width=.8cm, minimum height=.8cm, draw=black!60, rounded corners=.2cm},
    party/.style={circle, minimum size=.8cm, draw=black!60},
    func/.style={rectangle, draw=black, minimum width = .8cm, minimum height = .8cm},
    title/.style={rectangle, minimum width=.8cm, minimum height=.8cm},
    label/.style={rectangle, minimum width=.8cm, minimum height=.8cm},
    dummyicon/.style={circle, inner sep=0.05cm, minimum size=.8cm, draw=black!60, dotted, line width=.03cm},
    partyset/.style={circle, inner sep=0.05cm, minimum size=.8cm, draw=black!60, densely dashed, line width=.03cm},
    partyicon/.style={circle, inner sep=0.05cm, minimum size=.8cm, draw=black!60, fill=white},
    placeholder/.style={circle, inner sep=0.05cm, minimum size=.8cm},
    advicon/.style={circle, inner sep=0.05cm, minimum size=.8cm, draw=black!60, double},
    smallpartyicon/.style={circle, inner sep=0.05cm, minimum size=.08cm, draw=black!60},
    line/.style={densely dotted, line width=.02cm},
    functionalitybox/.style={draw=black!70,fill=black!70, text=white, minimum width=1cm,minimum height=.6cm}
}

%% Print URLs not in Typewriter Font
\def\UrlFont{\rm}

%% Page numbering
% We need page numbers according to https://crypto.iacr.org/2021/callforpapers.html
\pagestyle{plain}

\ifbeamer{
    \beamertemplatenavigationsymbolsempty
}{}

%% ulem...
\normalem

%% Equations alignment
\AtBeginDocument{\setlength\abovedisplayskip{4pt}}
\AtBeginDocument{\setlength\belowdisplayskip{4pt}}

%% Typography
\renewcommand*{\bibfont}{\small}

%% Table
\def\arraystretch{1.2}
\AtBeginDocument{\setlength{\tabcolsep}{0.5em}}

%% Default hyper setup
\ifanonymous
    \hypersetup{
        final,
        linktoc=all,
        pdfauthor={},
        pdftitle={},
        pdfsubject={},
        pdfkeywords={}
    }
\fi
\setcounter{tocdepth}{2}

%% Cross-referencing: hyper-xr
\makeatletter
\newcommand*{\addFileDependency}[1]{% argument=file name and extension
    \typeout{(#1)}
    \@addtofilelist{#1}
    \IfFileExists{#1}{}{\typeout{No file #1.}}
}
\makeatother

\newcommand*{\myexternaldocument}[1]{%
    \externaldocument{#1}%
    \addFileDependency{#1.tex}%
    \addFileDependency{#1.aux}%
}

%% providing llncs dummy commands
\providecommand{\inst}[1]{}
\providecommand{\institute}[1]{}
\providecommand{\email}[1]{}
\providecommand{\keywords}[1]{}
\providecommand{\and}{}

%% My TOC
\setcommand{\mytoc}{
    \begingroup
    \let\clearpage\relax
    \let\chapter\section
    \def\contentsname{Contents}
    \tableofcontents
    \endgroup
}

%% My maketitle
\setcommand{\mymaketitle}{
    \begingroup
    \let\oldaddcontentsline\addcontentsline
    \renewcommand*{\addcontentsline}[3]{%
        \ifstrequal{##1}{toc}{%
            \ifstrequal{##2}{author}{}{%
                \ifstrequal{##2}{title}{}{%
                    \oldaddcontentsline{##1}{##2}{##3}
                }
            }
        }{%
            \oldaddcontentsline{##1}{##2}{##3}
        }
    }
    %\let\addtocontents\relax
    \maketitle
    \endgroup
    % Don't have the word "Abstract" in the beginning
    \let\abstractname\relax
}

%% Auto delimiters
\let\ifdisplay\iffalse
\everydisplay\expandafter{\let\ifdisplay\iftrue}
\setcommand{\DeclareAutoPairedDelimiter}[3]{%
    \setcommand{#1}[1]{\ifdisplay\left#2\else#2\fi##1\ifdisplay\right#3\else#3\fi}
}

%% auto cref
\let\originalcref\cref
\setcommand{\cref}[1]{\ifdisplay\textup{\originalcref{#1}}\else\originalcref{#1}\fi}

%% strikeout in math mode
\let\originalsout\sout
\setcommand{\sout}[1]{\ifmmode\text{\originalsout{\ensuremath{#1}}}\else\originalsout{#1}\fi}
